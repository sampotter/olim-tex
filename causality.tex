\documentclass[eikonal.tex]{subfiles}

\begin{document}

I'm starting from:
\begin{equation}
  \frac{\hat{U} - U_0}{s^\theta h} = \sqrt{v^\top(I - QQ^\top)v} \sqrt{w^\top(I - QQ^\top)w} + v^\top QQ^\top w
\end{equation}
in your writeup, where $v$ and $w$ are $p_0$ and
$p_\lambda/l_\lambda$. The proof that leads to this equation can be
done with $p_i$ instead of $p_0$. Replacing $p_0$ with $p_i$ and
substituting $p_i$ and $p_\lambda/l_\lambda$ for $v$ and $w$ we get:
\begin{equation}
  \begin{split}\label{eq:2}
    \frac{\hat{U} - U_i}{s^\theta h} &= \frac{\sqrt{p_i^\top(I - QQ^\top)p_i} \sqrt{p_\lambda^\top(I - QQ^\top)p_\lambda} + p_i^\top QQ^\top p_\lambda}{l_\lambda} \\
    &= \frac{1}{l_\lambda} \parens{\norm{p_i}_{I - QQ^\top} \norm{p_\lambda}_{I - QQ^\top} + \langle p_i, p_\lambda \rangle_{QQ^\top}},
  \end{split}
\end{equation}
using the notation for weighted norms and inner products. Since
$I - QQ^\top$ projects onto $\range(\delta P)^\perp$, and since
$\dim \range(\delta P^\perp) = 1$, we can see that $(I - QQ^\top)p_i$
and $(I - QQ^\top)p_\lambda$ are multiples of one another---we can
also see that the scaling factor must be positive or zero (in which
case $\hat{p} \in \conv(p_0, \hdots, p_{n-1})$). By Cauchy-Schwarz,
this implies:
\begin{equation}
  \norm{p_i}_{I - QQ^\top} \norm{p_\lambda}_{I - QQ^\top} = \langle p_i, p_\lambda \rangle_{I - QQ^\top} = \langle p_i, p_\lambda \rangle - \langle p_i, p_\lambda \rangle_{QQ^\top}.
\end{equation}
Combining this with \cref{eq:2}, we get:
\begin{equation}
  \hat{U} - U_i = \frac{s^\theta h}{l_\lambda} p_i^\top p_\lambda.
\end{equation}
If we let $\mu_i(\lambda)$ be the cosine of the angle between $p_i$ and
$p_\lambda$, then this can be rewritten:
\begin{equation}
  \hat{U} - U_i = s^\theta h \norm{p_i} \mu_i(\lambda).
\end{equation}
Since $\mu_i(\lambda)$ is the only factor which can be negative, this
gives a very simple geometric sufficient condition for causality:
there can be no choice of $\lambda$ which creates an oblique angle
between $p_i$ and $p_\lambda$. Furthermore, since:
\begin{equation}
  p_i^\top p_\lambda = \lambda_0 p_i^\top p_0 + \cdots + \lambda_{n-1} p_i^\top p_{n-1},
\end{equation}
another sufficient condition is that all of the dot products
$p_i^\top p_j$ are nonnegative; equivalently, that all of the vectors
$p_i$ lie in a (possibly rotated) orthant. Since the tetrahedra of all
of our groups satisfy this condition, we can immediately see that the
corresponding update rules are causal.

To check the size of the gap for future Dial algorithm explorations,
we want to compute:
\begin{equation}
  \gamma = \min_i \hat{U} - U_i = s^\theta h \min_i \min_\lambda \norm{p_i} \mu_i(\lambda).
\end{equation}
This leads to a simple algorithm to compute $\gamma$: for each vertex
$p_i$, we want to find $p_\lambda$ which forms the largest angle with
$p_i$. Because of the geometry of the simplex, this will be another
vertex $p_j$. So, letting $\mu_{ij}$ be the cosine of the angle
between $p_i$ and $p_j$, we just compute:
\begin{equation}
  \gamma = s^\theta h \min_{i,j} \norm{p_i} \mu_{ij}.
\end{equation}
I did a few quick MATLAB tests and my results seem to agree with
yours.  By the way, one interesting consequence of this is that the
OLIM6 simplex won't work for a Dial-like solver. For each ``causal
simplex'', $\gamma \geq 0$ by definition: the OLIM6 simplex attains
$\gamma = 0$. I'm not sure if the sufficient condition concerning the
simplex lying in a rotated orthant is also necessary, but if you
imagine a simplex expanding so that its conical hull fills the orthant
(with the limiting case being OLIM6), then this would seem to suggest
that it is.

\end{document}

%%% Local Variables:
%%% mode: latex
%%% TeX-master: t
%%% End:
