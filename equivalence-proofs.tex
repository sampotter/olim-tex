\documentclass[sisc-eikonal.tex]{subfiles}

\begin{document}

\section{Proofs for section
  \ref{ssec:equivalence}}\label{sec:equivalence-proofs}

\begin{proof}[Proof of \cref{thm:equivalence}]
  We assume that $U$ is a linear function in the update simplex;
  hence, $\nabla U$ is constant. By stacking and subtracting
  \cref{eq:finite-differences} for different values of $i$, we obtain,
  for $i = 0, \hdots, n - 1$:
  \begin{equation}\label{eq:finite-diff-eq}
    \begin{bmatrix}
      \delP^\top \\
      p_i^\top
    \end{bmatrix} \nabla U = \begin{bmatrix}
      \delU \\
      U_0 - \hat{U}
    \end{bmatrix}.
  \end{equation}
  The inverse of the matrix in the left-hand side of \cref{eq:finite-diff-eq} is:
  \begin{equation}
    \begin{bmatrix}
      \parens{I - \frac{\numin p_i^\top}{\numin^\top p_i}} Q R^{-\top}, &
      \frac{\numin}{\numin^\top p_i}
    \end{bmatrix},
  \end{equation}
  which can be checked. This gives us:
  \begin{equation}
    \nabla U = \parens{I - \frac{\numin p_i^\top}{\numin^\top p_i}} Q R^{-\top} \delU + \frac{U_i - \hat{U}}{\numin^\top p_i} \numin.
  \end{equation}
  Hence, $\|\nabla U\|^2$ is a quadratic equation in
  $\hat{U} - U_i$. Expanding $\|\nabla U\|^2$, a number of
  cancellations occur since $Q^\top \numin = 0$. We have:
  \begin{equation}
    \delU^\top R^{-1} Q^\top \hspace{-0.25em} \parens{I - \frac{\numin p_i^\top}{\numin^\top p_i}}^{\hspace{-0.25em}\top} \hspace{-0.4em} \parens{I - \frac{\numin p_i^\top}{\numin^\top p_i}} \hspace{-0.1em} Q R^{-\top} \delU = \norm{R^{-\top} \delU}^2 + \frac{\parens{p_i^\top Q R^{-\top} \delU}^2}{\norm{\pmin}^2},
  \end{equation}
  so that, written in standard form:
  \begin{equation}
    \begin{aligned}
      {(\hat{U} - U_i)}^2 + 2 p_i^\top Q R^{-\top} \delU {(\hat{U} - U_i)} \,&+\, \parens{p_i^\top Q R^{-\top} \delU}^2 + \\
       &\norm{\pmin}^2 \parens{\norm{R^{-\top} \delU}^2 - \parens{s^\theta h}^2} = 0.
    \end{aligned}
  \end{equation}
  Solving for $\hat{U} - U_i$ gives:
  \begin{equation}
    \hat{U} = U_i - p_i^\top Q R^{-\top} \delU + \norm{\pmin} \sqrt{\parens{s^\theta h} - \|R^{-\top} \delU\|^2},
  \end{equation}
  establishing \cref{eq:U-finite-diff}.

  Next, to show that $\hat{U}' = \hat{U}$, we compute:
  \begin{align*}
    \hat{U}'
    &= U_0 + \delU^\top \lambda^* + s^\theta h l_{\lambda^*} & \cref{eq:f0-definition} \\
    &= U_0 - \parens{Q^\top p_0 + l_{\lambda^*} R^{-\top} \frac{\delU}{s^\theta h}}^\top R^{-\top} \delU + s^\theta h l_{\lambda^*} & \cref{eq:f0-exact-lambda} \\
    &= U_0 - p_0^\top Q R^{-\top} \delU + s^\theta h l_{\lambda^*} \parens{1 - \norm{R^{-\top} \frac{\delU}{s^\theta h}}^2} \\
    &= U_0 - p_0^\top Q R^{-\top} \delU + \norm{\pmin} \sqrt{\parens{s^\theta h}^2 - \norm{R^\top \delU}^2} = \hat{U}. & \cref{eq:l-star-expression}
  \end{align*}
  To establish \cref{eq:U-from-Ui-exact}, first note that
  $-R^{-\top} \delU = s^\theta h Q^\top \nu_{\lambda^*}$ by
  optimality. Substituting this into \cref{eq:U-finite-diff}, we first
  obtain:
  \begin{equation}
    \hat{U} = U_i + \frac{s^\theta h}{l_{\lambda^*}} \parens{p_i^\top \mathcal{P}_{\delP} p_{\lambda^*} + \norm{\pmin} \sqrt{p_{\lambda^*}^\top \mathcal{P}^\perp_{\delP} p_{\lambda^*}}}.
  \end{equation}
  Now, using the notation for weighted norms and inner products, we have:
  \begin{equation}\label{eq:weighted-norms}
    p_i^\top \mathcal{P}_{\delP} p_{\lambda^*} + \norm{\pmin} \sqrt{p_{\lambda^*}^\top \mathcal{P}^\perp_{\delP} p_{\lambda^*}} = \langle p_i, p_{\lambda^*} \rangle_{\mathcal{P}_{\delP}} + \norm{p_i}_{\mathcal{P}^\perp_{\delP}} \norm{p_{\lambda^*}}_{\mathcal{P}^\perp_{\delP}}.
  \end{equation}
  Since $\mathcal{P}^\perp_{\delP}$ orthogonally projects onto
  $\range(\delP)^\perp$, and since the dimension of this subspace
  is 1, $\mathcal{P}^\perp_{\delP} p_i$ and
  $\mathcal{P}^\perp_{\delP} p_{\lambda^*}$ are multiples of one
  another; furthermore, the angle between them is since our simplex is
  nondegenerate. So, by Cauchy-Schwarz:
  \begin{equation}\label{eq:cauchy-schwarz}
    \norm{p_i}_{\mathcal{P}^\perp_{\delP}} \norm{p_{\lambda^*}}_{\mathcal{P}^\perp_{\delP}} = \langle p_i, p_{\lambda^*} \rangle_{\mathcal{P}^\perp_{\delP}}.
  \end{equation}
  Combining \cref{eq:cauchy-schwarz} with \cref{eq:weighted-norms} and
  cancelling terms yields:
  \begin{equation}
    p_i^\top \mathcal{P}_{\delP} p_{\lambda^*} + \norm{\pmin} \sqrt{p_{\lambda^*}^\top \mathcal{P}_{\delP} p_{\lambda^*}} = p_i^\top p_{\lambda^*}.
  \end{equation}
  Equation \cref{eq:U-from-Ui-exact} follows.

  To parametrize the characteristic found by solving the finite
  difference problem, first note that the characteristic arriving at
  $\hat{p}$ is colinear with $\nabla \hat{U}$. If we let $\tilde{\nu}$
  be the normal pointing from $\hat{p}$ in the direction of the
  arriving characteristic, let $\tilde{p}$ be the point of
  intersection between $p_0 + \langle \delP \rangle$ and
  $\langle \tilde{\nu} \rangle$, and let
  $\tilde{l} = \norm{\tilde{p}}$, then, since
  $\tilde{p} - p_0 \in \langle \delP \rangle$:
  \begin{equation}
    \numin^\top (\tilde{p} - p_0) = 0.
  \end{equation}
  Rearranging this and substituting
  $\tilde{p} = \tilde{l} \tilde{\nu}$, we get:
  \begin{equation}
    \tilde{l} = \frac{\numin^\top p_0}{\numin^\top \tilde{\nu}}.
  \end{equation}
  Now, if we assume that we can write
  $\tilde{p} = \delP \tilde{\lambda} + p_0$ for some
  $\tilde{\lambda}$, then:
  \begin{equation}
    \tilde{\lambda} = R^{-1} Q^\top \parens{\tilde{p} - p_0} = -R^{-1} Q^\top \parens{I - \frac{\tilde{\nu} \numin^\top}{\tilde{\nu}^\top \numin}} p_0.
  \end{equation}

  To see that $\tilde{p} = p_{\lambda^*}$, note that since
  $\tilde{\nu} = -\nabla \hat{U}/\norm{\nabla \hat{U}} = -\nabla
  \hat{U}/(s^\theta h)$:
  \begin{equation}
    \mathcal{P}_{\delP} \tilde{\nu} = \frac{-\mathcal{P}_{\delP} \nabla \hat{U}}{s^\theta h} = \frac{-QR^{-\top} \delU}{s^\theta h} = \mathcal{P}_{\delP} \nu_{\lambda^*}.
  \end{equation}
  Since $\tilde{\nu}$ and $\nu_{\lambda^*}$ each lie in the unit
  sphere on the same side of the hyperplane spanned by $\delP$, and
  since $\mathcal{P}_{\delP}$ orthogonally projects onto
  $\range(\delP)$, we can see that in fact
  $\tilde{\nu} = \nu_{\lambda^*}$. Hence,
  $\tilde{p} = p_{\lambda^*} \in p_0 + \range(\delP)$. The second
  and third parts of \cref{thm:equivalence} follow.
\end{proof}

\end{document}

%%% Local Variables:
%%% mode: latex
%%% TeX-master: "sisc-eikonal.tex"
%%% End:
