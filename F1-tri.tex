\documentclass[eikonal.tex]{subfiles}

\begin{document}

\section{$F_1$ Triangle Updates}

\textbf{TODO}: we can't follow the same procedure we followed before:
otherwise we run into a quartic equation in $\lambda$ that we need to
solve. It's simpler to use Newton's method.

\begin{proof}
  Since
  $F_1(\lambda; \theta) = u(\lambda) + h s_\theta(\lambda)
  l(\lambda)$, we have:
  \begin{align*}
    F_1'(\lambda; \theta) = \delta u + h \parens{\theta \cdot \delta s \cdot l(\lambda) + \frac{s_\theta(\lambda) q'(\lambda)}{2 l(\lambda)}} = \delta u + \frac{h}{l(\lambda)} \parens{\theta \cdot \delta s \cdot p_\lambda + s_\theta(\lambda) \delta p}^\top p_\lambda.
  \end{align*}
  We also have:
  \begin{align*}
    F_1''(\lambda; \theta) = -\frac{h}{l(\lambda)} \parens{2 \theta \delta s p_\lambda^\top \delta p + s(\lambda; \theta) \delta p^\top \calP_{p_\lambda}^\perp \delta p}.
  \end{align*}
  This gives the rule:
  \begin{align*}
    x_{\operatorname{next}} \gets x + \frac{\langle \theta \delta s p_\lambda + s(\lambda; \theta) \delta p, p_\lambda \rangle}{\langle 2 \theta \delta s p_\lambda + s(\lambda; \theta) \calP^\perp_{p_\lambda} \delta p, \delta p \rangle}.
  \end{align*}
\end{proof}

\end{document}


%%% Local Variables:
%%% mode: latex
%%% TeX-master: t
%%% End:
