\documentclass[eikonal.tex]{subfiles}

\begin{document}

\subsection{Causality.}

\begin{theorem}
  ``The method is causal?'' (\textbf{TODO}: where is this
  defined...? why is it important?)
\end{theorem}

\begin{proof}
  Let $F = F_i$ for $i = 0, 1$.  Let
  $\hat{u} = \min_{\lambda \in \Delta^n} F(\lambda)$, i.e.
  $\hat{u} = F(\lambda^*) = u_{\lambda^*} + s^\theta h l_{\lambda^*}$.
  WTS: $\hat{u} \geq \max_{1 \leq i \leq n} u_i$. Assume WLOG that
  $\umax = u_0 = \max_{1\leq i \leq n} u_i$. Then, we equivalently want to
  prove:
  \begin{equation}
    c(\lambda^*) = \sum_{i=1}^n {(\umax - u_i)} \lambda^*_i = -\delta u^\top \lambda^* \leq s^\theta h l_{\lambda^*}
  \end{equation}
  In general, we must have:
  \begin{equation}
    c(\lambda^*) \leq \max_{\lambda \in \Delta^n} c(\lambda) \overset{(*)}{=} \max_{1 \leq i \leq n} c(e_i) = \umax - \min_{1\leq i \leq n} u_i = \umax - \umin \leq Ch
  \end{equation}
  for some $C > 0$. We justify the equality $(*)$ in the following
  way:
  \begin{itemize}
  \item The function $c(\lambda)$ is linear, and the set $\Delta^n$
    can be described by a linear matrix inequality; hence,
    $\max_{\lambda \in \Delta^n} c(\lambda)$ is a linear program and
    its optimum must be attained by one of its vertices.
  \item Clearly, $c(0) = 0$; furthermore, $\umax - u_i \geq 0$ for
    all $i$ such that $1 \leq i \leq n$.
  \item From these two observations, we can see that the optimum
    occurs at one of the vertices, and that ruling out the vertex
    $0 \in \Delta^n$ is safe, since $c(0)$ cannot be strictly greater
    than all other vertices $e_i \in \Delta^n$.
  \end{itemize}
  On the other hand, note that $l_\lambda$ is a strictly convex
  function; hence,
  \begin{equation}
    \min
  \end{equation}
\end{proof}

\end{document}

%%% Local Variables:
%%% mode: latex
%%% TeX-master: t
%%% End:
