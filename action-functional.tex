\documentclass[eikonal.tex]{subfiles}

\begin{document}

\section[Minimum action integral]{Minimum actional integral for the
  eikonal equation}\label{sec:minimum-action-integral} The eikonal
equation \cref{eq:eikonal} is a Hamilton-Jacobi equation
for $u$. If we let each fixed characteristic (ray) of the eikonal
equation be parametrized by some parameter $\sigma$ and denote
$p = x' = dx/d\sigma$, the corresponding Hamiltonian is:
\begin{equation}
  \label{eq:eikonal-hamiltonian}
  H{(p, x)} = \frac{\norm{p}^2}{2} - \frac{s(x)^2}{2} = 0.
\end{equation}
Since $H = 0$, \cref{eq:eikonal-hamiltonian} implies
$L = \sup_p (\langle p, x' \rangle - H) = s(x) \norm{x'}$. Since
$x' = \partial_p H = p$ and $\norm{p} = s(x)$, the Lagrangian is given
by:
\begin{equation}
  \label{eq:eikonal-lagrangian}
  L(x, x') = \langle p, x'\rangle = \langle x', x'\rangle = \norm{x'}^2,
\end{equation}
since $x' = p $. Hence, we can write:
\begin{equation}
  L(x, x') = \norm{x'}^2 = \abs{\langle \nabla u(x), x'\rangle} = \langle \nabla u(x), x'\rangle = u'.
\end{equation}

Let $x(\sigma)$ be a characteristic arriving at
$\hat{x} = x(\hat\sigma)$ from $x_0 = x(0)$, which lies on the
expanding front. Integrating from $0$ to $\hat\sigma$ and letting
$\hat u = u(\hat x)$ and $u_0 = u(x_0)$:
\begin{equation}
  \label{eq:minimum-action-on-a-ray}
  \hat u - u_0 = \int_{0}^{\hat\sigma} L(x, x') d\sigma = \int_{0}^{\hat\sigma} s(x) \norm{x'} d\sigma = \int_0^L s(x) dl,
\end{equation}
where $L$ is the length of the characteristic from $x_0$ to $\hat{x}$
and $dl$ is the length element. A characteristic of \cref{eq:eikonal}
minimizes \cref{eq:minimum-action-on-a-ray} if the path is allowed to
vary. Then, if $\hat{x}$ is fixed and $\alpha$ is an arc-length
parametrized curve with $\alpha(L) = \hat{x}$,
\cref{eq:minimum-action-on-a-ray} is equivalent to:
\begin{equation}\label{eq:eikonal-minimum-action-path}
  \hat{u} = u(\hat{x}) = \min_\alpha \curlyb{u(\alpha(0)) + \int_\alpha s(x) dl}.
\end{equation}
Our update procedure is based on
\cref{eq:eikonal-minimum-action-path}. This problem may have multiple
local minima---$\hat{u}$ above corresponds to the first arrival, which
is what interests us primarily in this work. Also, while the standard
finite difference method effectively discretizes the Hamiltonian, the
method presented here discretizes
\cref{eq:eikonal-minimum-action-path}. In this way, we can see that it
relates to the method of characteristics.

\end{document}

%%% Local Variables:
%%% mode: latex
%%% TeX-master: "sisc-eikonal.tex"
%%% End:
