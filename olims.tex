\documentclass[eikonal.tex]{subfiles}

\begin{document}

\section{Ordered Line Integral Methods}

To justify the present work, we start by presenting the update
procedure (step 6 of the fast marching method described in
\cref{sec:intro}) for the standard fast marching method, which is
based on finite differences. We then present the update for a
prototypical version of an \emph{ordered line integral method} (OLIM)
and show that it is equivalent to the standard fast marching method's
update. The rest of the work will then concern itself with
improvements and refinements of our first OLIM.

Let $h > 0$, and assume our grid forms a subset of $h
\mathbb{Z}^{n+1}$. Let $\hat{p} \in h \mathbb{Z}^{n+1}$ be the node with the
minimal heap value (as in step 3 in the informal description of the
algorithm); and, without loss of generality, identify $\hat{p}$ with
the origin. During the update step for the fast marching method, the
valid neighbors with the minimum value along each axis are
extracted. For example, for each $j$ such that $0 \leq j \leq n$, one
of $\pm h e_j$ is selected, depending on the values of $U$ and whether
node is \texttt{valid}. There may be fewer than $n$ nodes selected
this way. Without loss of generality, we assume that $n$ nodes have
been selected, and that they are given by $\{h e_i\}_{i=0}^n$. Once
applied, the update rule consists of solving the following equation
for $\hat{u}$:
\begin{equation}
  \label{eq:fmm-equation}
  \sum_{i=0}^n {(\hat{u} - u_i)}^2 = \hat{s}^2 h^2.
\end{equation}
The value of $\hat{u}$ is given by:
\begin{equation}
  \label{eq:fmm-uhat}
  \hat{u} = \frac{1}{n} \sum_{i=0}^n u_i + \frac{1}{n} \sqrt{\parens{\sum_{i=0}^n u_j}^2 - n \parens{\sum_{i=0}^n u_i^2 - \hat{s}^2 h^2}}.
\end{equation}
If the discriminant of \cref{eq:fmm-uhat} is negative, a
lower-dimensional update is performed a subset of the selected nodes.

We will consider the OLIMs in $\R^{n+1}$ with the same stencil as the
(finite difference-based) fast marching method---in particular, where
the stencil for each node consists of its $2(n+1)$ nearest
neighbors---and using the right-hand rule, $\Frhr$. From
\cref{eq:hat-u-variational}, we solve:
\begin{equation}
  \label{eq:basic-olim-min}
  \hat{u} = \min_{\lambda \in \Delta^n} \Frhr(\lambda) = \min_{\lambda \in \Delta^n} \set{u_\lambda + \hat{s} h l_\lambda}.
\end{equation}
where we write $s^\theta = s^0 = \hat{s}$ since $\hat{s}$ corresponds
with $s$ evaluated at $\hat{p}$. For the fast marching marching method
update, the method of selecting which nodes incorporate into
\cref{eq:fmm-equation} is determined by the discretization of the
differential operator. The ordered line integral method is akin to the
method of characteristics, and the considerations become geometric in
nature. \textbf{TODO}: say more about this.

\begin{theorem}
  As before, without loss of generality, let the update nodes be
  $\{h e_i\}_{i=0}^n$. Then, the solution of \cref{eq:fmm-equation}
  corresponds with the solution of \cref{eq:basic-olim-min},
  \textbf{TODO}: under certain conditions.
\end{theorem}

\begin{proof}
  First, we solve \cref{eq:basic-olim-min} by the method of Lagrange
  multipliers. Ignoring the requirement that $\lambda_i \geq 0$ for
  $i$ such that $0 \leq i \leq n$, we solve \cref{eq:basic-olim-min}
  only subject to the requirement that
  $g(\lambda) = \m{1}^\top \lambda - 1 = 0$. Defining the Lagrangian
  $L(\lambda, \alpha) = \Frhr(\lambda) - \alpha g(\lambda)$, the
  resulting system of equations is
  $\alpha \m{1} - u = \hat{s} h \lambda / l_\lambda$ and
  $\m{1}^\top \lambda = 1$. Squaring the first of these gives us
  $\norm{\alpha \m{1} - u}^2 = \hat{s}^2 h^2$, where the dependence on
  $\lambda$ drops out. Furthermore, letting $(\lambda^*, \alpha^*)$ be
  optimum, if we multiply $\nabla_\lambda L(\lambda^*, \alpha^*) = 0$ through by
  $\lambda^*$, we obtain:
  \begin{equation*}
    0 = \nabla_\lambda L(\lambda^*, \alpha^*)^\top \lambda^* = u^\top \lambda^* - \alpha^* \m{1}^\top \lambda^* + \hat{s} h \frac{l_{\lambda^*}^2}{l_{\lambda^*}} = u_\lambda^* - \alpha^* + \hat{s} h l_{\lambda^*}.
  \end{equation*}
  Hence, $\alpha^* = u_{\lambda^*} + \hat{s} h l_{\lambda^*} = \hat{u}$, and
  $\norm{\hat{u} \m{1} - u}^2 = \hat{s}^2 h^2$, recovering
  \cref{eq:fmm-equation}.
\end{proof}

\end{document}

%%% Local Variables:
%%% mode: latex
%%% TeX-master: t
%%% End:
