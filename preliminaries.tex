\documentclass[eikonal.tex]{subfiles}

\begin{document}

\section{Notation}

In this section, we briefly present some notation and definitions that
will be used throughout. Let $n \geq 2$ be an integer and assume
$d \in \set{0, \hdots, n - 1}$. Let
$\set{p_0, \hdots, p_d} \subseteq \R^n$ be a linearly independent set
of vectors (implying $d < n$). We will be interested in convex
combinations of these vectors. Define the set $\Delta^d$ by:
\begin{align*}
  \Delta^d = \set{(\lambda_1, \hdots, \lambda_d) : \lambda_i \geq 0 \mbox{ for } i = 1, \hdots, d \mbox{ and } \sum_{i=1}^d \lambda_i \leq 1}.
\end{align*}
For each $\lambda = (\lambda_1, \hdots, \lambda_d) \in \Delta^d$, we
will write $\lambda_0 = 1 - \lambda_1 - \cdots - \lambda_d$. Then,
$\sum_{i=0}^d \lambda_i p_i$ lies in the convex hull of
$\set{p_0, \hdots, p_d}$. We will typically write this vector
$p_\lambda$. If we define $\delp_i = p_i - p_0$, we can also write
$p_\lambda = p_0 + \sum_{i=1}^d \lambda_i \delp_i$. This is a
homogeneous transformation applied to $\lambda$; to this end, we
define the matrix:
\begin{align*}
  \delP = \begin{pmatrix} \delp_1 & \cdots & \delp_d \end{pmatrix} \in \R^{n \times d},
\end{align*}
so that we can write $p_\lambda = \delP \lambda + p_0$. We will
have reason to make use of this ``$\delta$'' notation elsewhere---in
general, if $x$ is some indexed object, then $\delta x_i = x_i - x_0$.

\end{document}

%%% Local Variables:
%%% mode: latex
%%% TeX-master: t
%%% End:
