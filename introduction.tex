\documentclass[eikonal.tex]{subfiles}

\begin{document}

\section{Introduction}\label{sec:intro}

\paragraph{The eikonal equation.} We are interested in solving the
\emph{eikonal equation}, a nonlinear PDE encountered in wave
propagation and the modeling of a wide variety of problems in
computational and applied science (\hl{cite}). With $n \geq 2$, and
given a domain $\Omega \in \R^n$, the eikonal equation is of the form:
\begin{equation}\label{eq:eikonal}
  \abs{\nabla u(x)} = s(x), \qquad x \in \Omega,
\end{equation}
where $s : \Omega \to \R_+$ is a fixed, nonnegative \emph{slowness
  function}, which forms part of the problem data. Hence, we solve for
$u : \Omega \to \overline{\R}_+$. The rest of the problem data is a
subset of $D \subset \Omega$ where $u$ has been fixed; i.e.,
$\left. u \right|_D = g$, for some $g : D \to \overline{\R}_+$. As an
example, if $s \equiv 1$ and $g \equiv 0$, then the solution $u$ of
\cref{eq:eikonal} is the distance to $D$ at each point in $\Omega$:
\begin{equation}
  \label{eq:distance-to-Omega}
  u(x) = d(x, D) = \inf_{y \in \Omega} \norm{x - y}_2.
\end{equation}
The rapid solution of this problem on complicated geometries is an
example of an exceedingly practical application of the fast marching
method, which sees use in areas such as computer graphics,
computational geometry, computed-aided design, and the like
(\hl{cite}).

\paragraph{The fast marching method.} As we have just alluded to, the
fast marching method is a particularly efficient method for solving
the eikonal equation~\cite{sethian1999level}. To describe this method,
first let $\calP = \{p_i\}_{i \in I} \subseteq \Omega$ be a set of
\emph{nodes} where we would like to approximate the true solution $u$
with a numerical solution $U : \calP \to
\overline{\mathbb{R}}_+$. Additionally, for each node $p \in \calP$,
define a set of neighbors,
$\neib(p) \subseteq \calP \backslash \set{p}$. The standard fast
marching method takes $\calP$ to be a lattice in $\R^n$ and $\neib(p)$
to be each node's $2n$ neighbhors. With $\calP$ defined, we also
define the \emph{boundary nodes}, the set $\boundary$. The set
$\boundary$ and $D$ may not coincide; to reconcile this difference,
the initial value of $U(p)$ for each $p \in \boundary$ must take $g$
into account in the best way possible.

We will give a brief description of the fast marching method. There
are several extra pieces of information that need to be kept track of
in order to implement the algorithm. For each node $p$, apart from the
current value of $U(p)$, the most salient piece of information is the
\emph{state} of each node $p$, written $p$\texttt{.state}
$\in \set{\texttt{valid}, \texttt{trial}, \texttt{far}}$. The meaning
of each of these states will become clear from the following
high-level description of the algorithm:

\begin{center}
  \begin{enumerate}[nolistsep]
  \item For each $p \in \calP$, initially set $p$\texttt{.state} $=$
    \texttt{far} and $U(p) = \infty$.
  \item For each $p \in \boundary$, set $p$\texttt{.state} $=$
    \texttt{valid}, initialize $U(p)$, and execute
    \cref{enum:set-trial,enum:update-U}.
  \item While there are \texttt{trial} or \texttt{far} nodes left:
    \begin{enumerate}[nolistsep]
    \item Let $p$ be the \texttt{trial} node with the smallest value
      $U(p)$.
    \item Set $p$\texttt{.state} $\gets$ \texttt{valid}.
    \item For each $q \in \neib(p)$, set $p$\texttt{.state} $\gets$
      \texttt{trial} if $p$\texttt{.state} $=$
      \texttt{far}.\label{enum:set-trial}
    \item For each $q \in \neib(\hat{p})$ such that $q$\texttt{.state}
      $=$ \texttt{trial}, update $U(q)$.\label{enum:update-U}
    \end{enumerate}
  \end{enumerate}
\end{center}

The exact sequencing and details of some of these steps is
variable. Our intent here is to give a birds-eye view of the
algorithm---for concrete details, a suitable reference should be
consulted (\hl{cite}).

The preceding algorithm is generic in the following ways:
\begin{itemize}
\item There are a variety of ways to compute $\boundary$ and
  subsequently approximate the initial value of $U$ for
  $p \in \boundary$ using $g$ (\hl{cite}).
\item The method of keeping track of the node with the smallest value
  is variable: most frequently, a binary heap storing pointers to the
  nodes is dynamically updated, leading to $O(N \log N)$ update
  operations, where $N$ is the number of nodes.
\item The update procedure itself can take different forms: for
  methods based on finite differences, there are first-order methods
  and ``almost'' higher-order methods (methods which take advantage of
  second- and higher-order discretizations when the nodes are
  available for them, and which default the lower order schemes
  otherwise).
\item Related to the foregoing point, the arrangement of the nodes
  (into a grid or otherwise) varies; hence, the neighborhood of each
  node varies. This naturally affects the update procedure.
\end{itemize}

\paragraph{Minimum action integral of the eikonal equation.} The
eikonal equation \cref{eq:eikonal} is a Hamilton-Jacobi equation for
$u$. If we let each fixed characteristic (ray) of the eikonal equation
be parametrized by some parameter $\sigma$ and denote
$p = x' = dx/d\sigma$, the corresponding Hamiltonian is:
\begin{equation}
  \label{eq:eikonal-hamiltonian}
  H{(p, x)} = \frac{\norm{p}^2}{2} - \frac{s(x)^2}{2} = 0.
\end{equation}
Since $H = 0$, \cref{eq:eikonal-hamiltonian} implies
$L = \langle p, x'\rangle = H$, where $x' = \partial_p H = p$.  From
Hamilton's equation, $H(p, x) = \langle p, x'\rangle - L(x, x')$, we
can see that the Lagrangian is given by:
\begin{equation}
  \label{eq:eikonal-lagrangian}
  L(x, x') = \langle p, x'\rangle = \langle x', x'\rangle = \norm{x'}^2,
\end{equation}
i.e. $L = \norm{x'}^2$. From \cref{eq:eikonal-hamiltonian}, we can
also see that $s(x) = \norm{p} = \norm{x'}$, since $s$ was assumed to
be nonnegative. Substituting this into \cref{eq:eikonal-lagrangian},
we get $L(x, x') = s(x) \norm{x'}$. By \cref{eq:eikonal-hamiltonian},
we have $x' = p = \nabla u(x)$. Since $L > 0$, we can write:
\begin{equation}
  L(x, x') = \norm{x'}^2 = \abs{\langle \nabla u(x), x'\rangle} = \langle \nabla u(x), x'\rangle = u'.
\end{equation}

Let $x(\sigma)$ be a characteristic arriving at
$\hat{x} = x(\hat\sigma)$ from $x_0 = x(0)$, which lies on the
expanding front. Integrating from $0$ to $\hat\sigma$ and letting
$\hat u = u(\hat x)$ and $u_0 = u(x_0)$:
\begin{equation}
  \label{eq:minimum-action-on-a-ray}
  \hat u - u_0 = \int_{0}^{\hat\sigma} L(x, x') d\sigma = \int_{0}^{\hat\sigma} s(x) \norm{x'} d\sigma = \int_0^L s(x) dl,
\end{equation}
where $L$ is the length of the characteristic from $x_0$ to $\hat{x}$
and $dl$ is the length element. A characteristic of \cref{eq:eikonal}
minimizes \cref{eq:minimum-action-on-a-ray} if the path is allowed to
vary. Then, if $\hat{x}$ is fixed and $\alpha$ is an arc-length
parametrized curve with $\alpha(L) = \hat{x}$,
\cref{eq:minimum-action-on-a-ray} is equivalent to:
\begin{equation}\label{eq:eikonal-minimum-action-path}
  \hat{u} = u(\hat{x}) = \min_\alpha \curlyb{u(\alpha(0)) + \int_\alpha s(x) dl}.
\end{equation}
Our update procedure is based on
\cref{eq:eikonal-minimum-action-path}. This problem may have multiple
local minima---$\hat{u}$ above corresponds to the first arrival, which
is what interests us primarily in this work. Also, while the standard
finite difference method effectively discretizes the Hamiltonian, the
method presented here discretizes
\cref{eq:eikonal-minimum-action-path}. In this way, we can see that it
relates to the method of characteristics.

\end{document}

%%% Local Variables:
%%% mode: latex
%%% TeX-master: t
%%% End:
