\documentclass[eikonal.tex]{subfiles}

\begin{document}

\section{Introduction}\label{sec:introduction}

In this work, we are interested in solving the \emph{eikonal
  equation}, a nonlinear hyperbolic PDE encountered in wave
propagation and the modeling of a wide variety of problems in
computational and applied
science~\cite{sethian1999level}. \hl{\textbf{TODO}}: \emph{add more
  background.}

\subsection{The eikonal equation}

With $n \geq 2$, and given a domain $\Omega \in \R^n$, the eikonal
equation is of the form:
\begin{equation}\label{eq:eikonal}
  \norm{\nabla u(x)}_2 = s(x), \qquad x \in \Omega,
\end{equation}
where $s : \Omega \to \R_+$ is a fixed, nonnegative \emph{slowness
  function}, which forms part of the problem data. Hence, we solve for
$u : \Omega \to \overline{\R}_+$. The rest of the problem data is a
subset $D \subset \Omega$ where $u$ has been fixed; i.e.,
$\left. u \right|_D = g$ for some $g : D \to \overline{\R}_+$. As an
example, if $s \equiv 1$ and $g \equiv 0$, then the solution $u$ of
\cref{eq:eikonal} is the distance to $D$ at each point in $\Omega$:
\begin{equation}
  \label{eq:distance-to-Omega}
  u(x) = d(x, D) = \inf_{y \in \Omega} \norm{x - y}_2.
\end{equation}
A variety of numerical methods have been proposed for the solution of
\cref{eq:eikonal}: most methods can be described as either
Dijkstra-like or fast sweeping methods, each of which is described
below.

\subsection{Results}

We develop a suite of fast Dijkstra-like solvers for the eikonal
equation in 2D and 3D, similar to the fast marching method or ordered
upwind methods~\cite{sethian1996fast,sethian2003ordered}. These
solvers come about by discretizing and minimizing the action
functional for the eikonal equation, which is a line integral. We
investigate three different quadrature rules for discretizing this
line integral: a righthand rule (\texttt{rhr}), a simplified midpoint
rule (\texttt{mp0}), and a midpoint rule (\texttt{mp1}). We also
consider ways of decomposing a grid point's neighborhood into
triangles and tetrahedra, which help us simplify and accelerate our
solvers by avoiding redundant and unnecessary computations. We extend
this algorithm to handle the locally factored eikonal equation using
an additive factorization~\cite{luo2012fast, qi2018corner}.

In 3D, to minimize the number of FLOPs required to solve
\cref{eq:eikonal}, we develope two separate hierarchical
algorithms. In the first, the solution of a system of nonlinear
equations is used to provide a warm start to a Newton iteration; this
is combined with a search for updates that starts with low-dimensional
updates and proceeds to higher ones, using information from
lower-dimensional updates to structure the search. In the second
algorithm, we come up with a way to quickly enumerate neighboring
simplexes and solve a collection of sequential quadratic programs
(SQPs) over the base of each. We provide theoretical results
establishing the validity of these methods and demonstrate their
interrelation. We conduct extensive numerical experiments, which
demonstrate that our solvers exceed the performance of existing
Dijkstra-like solvers in terms of error measured versus CPU time
(i.e., for a fixed error tolerance, our solvers obtain a sufficiently
accurate solution in less time). \hl{\textbf{TODO}}: \emph{add the
  speedup factor here once we know it.}

\end{document}

%%% Local Variables:
%%% mode: latex
%%% TeX-master: "sisc-eikonal.tex"
%%% End:
