\documentclass[eikonal.tex]{subfiles}

\begin{document}

\section{Introduction}\label{sec:introduction}

In this work, we are interested in solving the \emph{eikonal
  equation}, a nonlinear hyperbolic PDE encountered in high-frequency
wave propagation~\cite{engquist2003computational} and the modeling of
a wide variety of problems in computational and applied
science~\cite{sethian1999level}, such as photorealistic
rendering~\cite{ihrke2007eikonal}, constructing signed distance
functions in the level set method~\cite{osher2006level}, solving the
shape from shading
problem~\cite{kimmel2001optimal,prados2006shape,durou2008numerical},
and others. We are motivated primarily by problems in high-frequency
acoustics~\cite{prislan2016ray}, which are key to enabling a higher
degree of verisimilitude in virtual reality simulations
(see~\cite{raghuvanshi2014parametric,raghuvanshi2018parametric} for a
cutting-edge time-domain approach); higher accuracy leads to more
faithful results, and fast algorithms allow for more economical use.

\subsection{Results}

Different numerical methods have been proposed for the solution of the
eikonal equation; generally, there are direct solvers and iterative
solvers. The most popular direct solvers are based on Dijkstra's
algorithm (we refer to these as ``Dijkstra-like'' from now on), and
the most popular iterative method is the fast sweeping method. In this
work, we develop a family of Dijkstra-like solvers for the eikonal
equation in 2D and 3D, similar to the fast marching method (FMM) or
ordered upwind
methods~\cite{sethian1996fast,sethian2003ordered}. These solvers come
about by discretizing and minimizing the action functional for the
eikonal equation (see \cref{sec:minimum-action-integral}). We
investigate three different quadrature rules for discretizing this
line integral: a righthand rule (\texttt{rhr}), a simplified midpoint
rule (\texttt{mp0}), and a midpoint rule (\texttt{mp1}). We also
consider different ways of organizing a grid point's neighborhood into
triangles and tetrahedra, simplifying and accelerating our solvers by
avoiding redundant and unnecessary computations, particularly in
3D. Additionally, we modify our algorithm to solve the additively
factored eikonal equation~\cite{luo2012fast}; along these lines, for
our numerical experiments, we follow a recently introduced
approach~\cite{qi2018corner} and locally factor point sources in order
to recover $O(h)$ convergence.

Our main results are as follows:
\begin{itemize}
\item For 3D problems, we develop two separate algorithms: a
  \emph{bottom-up} (\texttt{olimhu}) algorithm, and a \emph{top-down}
  algorithm (\texttt{olim\emph{N}}, where \texttt{\emph{N}}
  \hspace{-0.1em}$=6,18,26$ is the size of neighborhood used). Each
  algorithm locally updates a grid point by performing a minimal
  number of triangle or tetrahedron updates. Depending on the
  quadrature rule, each update is calculated by solving a system of
  nonlinear equations either directly (\texttt{rhr} and \texttt{mp0})
  or iteratively (\texttt{mp1}).
\item We note that this work was done in tandem with work constructing
  ordered line integral methods for computing the quasipotential for
  nongradient stochastic differential equations
  (SDEs)~\cite{dahiya2017ordered,yang2018computing,dahiya2018ordered}. Unlike
  the quasipotential, the eikonal equation is simple enough to allow
  us to analyze and justify our algorithms. We show how our quadrature
  rules relate to one another, rigorously justifying the \texttt{mp0}
  rule, and establishing it as preferable to \texttt{mp1} in most
  cases.
\item We conduct extensive numerical tests on a variety of problems,
  including point source problems for different slowness (index of
  refraction) functions, and multiple point source problems with a
  linear speed function. These problems have analytical solutions,
  which we use as a ground truth for comparison.
\item We show how our algorithms relate to the standard FMM and
  compare runtimes, numerically showing that only a modest slow-down
  is incurred by the version of our top-down algorithm which is
  equivalent to the FMM (specifically, \texttt{olim6} with the
  \texttt{rhr} quadrature rule is computes the same result as the
  FMM). \textbf{TODO}: include the final \% slowdown.
\end{itemize}

\end{document}

%%% Local Variables:
%%% mode: latex
%%% TeX-master: "sisc-eikonal.tex"
%%% End:
