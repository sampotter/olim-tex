\documentclass{article}

\usepackage{amsmath,amssymb,amsthm,cleveref,fullpage,nicefrac}

\newcommand{\Fmpzero}{F_0^{\operatorname{(mp)}}}
\newcommand{\Fmpone}{F_1^{\operatorname{(mp)}}}
\newcommand{\Frhr}{F^{\operatorname{(rhr)}}}
\newcommand{\calC}{\mathcal{C}}
\newcommand{\norm}[1]{\left|\left|#1\right|\right|}
\newcommand{\set}[1]{\left\{#1\right\}}
\newcommand{\squareb}[1]{\left[#1\right]}
\newcommand{\parens}[1]{\left(#1\right)}

\begin{document}

\section{Quadrature Rules}

The simplex updates of each OLIM approximately minimize a line
integral which corresponds a trial evolution of the solution's
characteristic curve. When performing a simplex update of dimension
$m$ for $C = \in \calC$, we have access to the numerical solution $u$
at the points $p_i \in v(C) \backslash \{\hat{p}\}$; i.e.,
$u_i = u(p_i)$ for $i = 0, \hdots, m$. Defining
$u_\lambda = \lambda_0 u_0 + \cdots + \lambda_m u_m$ and
$p_\lambda = \lambda_0 p_0 + \cdots + \lambda_m p_m$, the line
integral to minimize is:
\begin{align}\label{eq:hat-u-variational}
  \hat{u} = \min_{\lambda \in \Delta^m} \set{u_\lambda + \int_{[p_\lambda, \hat{p}]} s(\gamma(t))dt}.
\end{align}
That is, we minimize the linear approximation $u_\lambda$ plus the
time taken to travel from the point $p_\lambda$ on the front of the
solution to the update point $\hat{p}$. In the integral above, we
assume that $\gamma(t)$ is an arc length parametrization of the
interval $[p_\lambda, \hat{p}]$.

In this work, we consider two approximations to
\cref{eq:hat-u-variational}: the difference between the two is the way
we incorporate the speed function $s$. Let $\theta \in [0, 1]$ and
define:
\begin{align*}
  F_0(\lambda; \theta) &= u_\lambda + \squareb{(1-\theta)\hat{s} + \frac{\theta}{m} \sum_{i=1}^m s_i} \norm{\hat{p} - p_\lambda}_2, \\
  F_1(\lambda; \theta) &= u_\lambda + \squareb{(1-\theta)\hat{s} + \theta s_\lambda} \norm{\hat{p} - p_\lambda}_2.
\end{align*}
We will primarily concern ourselves with $F_0$ and $F_1$ for
$\theta = 0$ and $\theta = \nicefrac{1}{2}$. For $\theta = 0$, we have
$F_0 = F_1$, so we define $\Frhr = F_0 = F_1$. On the other hand,
$F_0(\lambda;\nicefrac{1}{2}) \neq F_1(\lambda;\nicefrac{1}{2})$
unless $s \equiv 1$, so we write
$\Fmpzero(\lambda) = F_0(\lambda; \nicefrac{1}{2})$ and
$\Fmpone(\lambda) = F_1(\lambda; \nicefrac{1}{2})$, where ``mp''
stands for ``midpoint''. We will define
$l_\lambda = \norm{\hat{p} - p_\lambda}/h$, so that:
\begin{align*}
  \Frhr(\lambda) = u_\lambda + \hat{s} h l_\lambda, \qquad \Fmpzero(\lambda) = u_\lambda + \frac{\hat{s} + \langle s \rangle}{2} h l_\lambda, \qquad \Fmpone(\lambda) = u_\lambda + \frac{\hat{s} + s_\lambda}{2} h l_\lambda.
\end{align*}

\end{document}

%%% Local Variables:
%%% mode: latex
%%% TeX-master: t
%%% End:
