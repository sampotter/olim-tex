\documentclass{article}

\usepackage{eikonal}

\begin{document}

Throughout the following, we will assume that the following quantities
are defined:
\begin{align*}
  \hat{p} &:= \text{the update point, taken to be zero for simplicity} \\
  p_0, p_1, p_2 &:= \text{the other three vertices of the update tetrahedra} \\
  h &:= \text{the lattice constant of the computational mesh} \\
  s &:= \text{a nonnegative slowness function} \\
  \hat{u} = u(\hat{p}) &:= \text{the numerical solution at $\hat{p}$} \\
  u_0, u_1, u_2 &:= \text{the computed numerical solution at $p_0, p_1$, and $p_2$} \\
  \lambda = (\lambda_1, \lambda_2) &:= \text{barycentric coordinates parametrizing the points $p_\lambda = p_0 + \lambda_1 (p_1 - p_0) + \lambda_2 (p_2 - p_0)$}
\end{align*}
For convenience, we will also define:
\begin{align*}
  \delU_i = u_i - u_0, \qquad \delp_i = p_i - p_0, \qquad u_\lambda = u_0 + \lambda_1 \delU_1 + \lambda \delU_2.
\end{align*}
In particular, $u_\lambda$ denotes a linear approximation to $u$ on
the triangle with $p_0, p_1$, and $p_2$ as its vertices.

\begin{lemma}
  Let $F(\lambda) = u_\lambda + \hat{s} ||\hat{p} - p_\lambda||$,
  define $\alpha_i = |\delU_i/(\hat{s} h)|$, and let:
  \begin{align*}
    A_i = 2 - \alpha_i^2, \qquad B_i = 1 - 2\alpha_i^2, \qquad C_i = 1 - \alpha_i^2,
  \end{align*}
  for $i = 1, 2$. Further, define:
  \begin{align*}
    Q_1(\lambda) &= 2A_1\lambda_1^2 + 2A_1\lambda_1\lambda_2 + B_1\lambda_2^2 -2A_1\lambda_1 -2C_1\lambda_2 + C_1, \\
    Q_2(\lambda) &= B_2\lambda_1^2 + 2A_2\lambda_1\lambda_2 + 2A_2\lambda_2^2 -2C_2\lambda_1 -2A_1\lambda_2 + C_2.
  \end{align*}
  Then, computing
  $\lambda^* = \operatorname{argmin}_{\lambda \in \Delta^2}
  F(\lambda)$ is equivalent to finding
  $\lambda^* \in \Delta^2 \cap Q_1 \cap Q_2$ such that:
  \begin{align*}
    \begin{bmatrix}
      2 & 1 \\ 1 & 2
    \end{bmatrix} \begin{bmatrix}
      \lambda_1 \\ \lambda_2
    \end{bmatrix} = \boldsymbol{1}_{2 \times 1} - \frac{l_\lambda}{\hat{s} h} \begin{bmatrix}
      \delU_1 \\ \delU_2
    \end{bmatrix}.
  \end{align*}
\end{lemma}

\end{document}

%%% Local Variables:
%%% mode: latex
%%% TeX-master: t
%%% End:
