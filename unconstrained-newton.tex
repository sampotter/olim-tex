\documentclass{standalone}

\usepackage{eikonal}

\begin{document}

\section{Unconstrained Minimization of $F_0$ and $F_1$}

\begin{lemma}
  Let $\set{\delp_1, \cdots, \delp_d, p_0}$ be linearly
  independent. Then:
  \begin{align*}
    \nabla_\lambda^2 F_0(\lambda; \theta)^\dagger \nabla F_0(\lambda; \theta) = \nabla_\lambda^2 F_0(\lambda; \theta)^\dagger \delU.
  \end{align*}
\end{lemma}

\begin{proof}
  Using \cref{prop:F0-grad-and-Hess} and the linear independence
  of $\set{\delp_1, \cdots, \delp_2, p_0}$:
  \begin{align*}
    \nabla^2_\lambda F_0(\lambda; \theta)^\dagger \nabla_\lambda F_0(\lambda; \theta) = \frac{l_\lambda}{s_\theta h} \delP^\dagger \parens{I - \frac{p_\lambda p_\lambda^\top}{p_\lambda^\top p_\lambda}}^\dagger \parens{\delP^{\top}}^\dagger \parens{\delU + \frac{s_\theta h}{l_\lambda} \delP^\top p_\lambda}.
  \end{align*}
  Since $I - p_\lambda p_\lambda^\top/(p_\lambda^\top p_\lambda)$ is a
  projector, it is its own pseudoinverse. Then, again using linear
  independence:
  \begin{align*}
    \parens{I - \frac{p_\lambda p_\lambda^\top}{p_\lambda^\top p_\lambda}} \parens{\delP^{\top}}^\dagger \delP^\top p_\lambda = 0,
  \end{align*}
  from which we can conclude the result.
\end{proof}

\begin{lemma}
  Let $g = \nabla^2_\lambda F_0(\lambda; \theta)^\dagger \delU$ and
  let $\beta_\lambda$ be given by:
  \begin{align*}
    \beta_\lambda = \frac{1}{\parens{s_\theta h}^2} \norm{\parens{\delP^\top}^\dagger \delU}^2_{\calP^\perp_{p_\lambda}}.
  \end{align*}
  Then, $t^* = \arg\min_{t \leq 0} \nabla F_0(\lambda + t g; \theta)$
  is given by $t^* = -1/\sqrt{1 - \beta_\lambda}$.
\end{lemma}

\begin{proof}
  Then:
  \begin{align}\label{eq:F0-line-deriv}
    \frac{d}{dt} \curlyb{F_0\parens{\lambda + t g; \theta}} &= \parens{\delU + \frac{s_\theta h}{l_{\lambda + t g}} \delP^\top p_{\lambda + t g}}^\top \frac{l_\lambda}{s_\theta h} \delP^\dagger \calP^\perp_{p_\lambda} \parens{\delP^\top}^\dagger \delU.
  \end{align}
  Since $p_{\lambda + tg} = p_\lambda + t \delP g$, we have:
  \begin{align}\label{eq:plam_t_g_simplified}
    p_{\lambda + t g}^\top \delP \delP^\dagger \mathcal{P}^\perp_{p_\lambda} &= \parens{p_\lambda + t \delP g}^\top \mathcal{P}^\perp_{p_\lambda} = t g^\top \delP^\top \calP^\perp_{p_\lambda} \\
                                                                                   &= \frac{t l_\lambda}{s_\theta h} \delU^\top \delP^\dagger \calP^\perp_{p_\lambda} \parens{\delP^\top}^\dagger \delP^\top \calP^\perp_{p_\lambda} = \frac{t l_\lambda}{s_\theta h} \squareb{\parens{\delP^\top}^\dagger \delU}^\top \calP^\perp_{p_\lambda}.
  \end{align}
  Writing
  $[(\delP^\top)^\dagger \delU]^\top \calP^\perp_{p_\lambda}
  (\delP^\top)^\dagger \delU = ||(\delP^\top)^\dagger \delta
  u||^2_{\calP^\perp_{p_\lambda}}$, and applying
  \cref{eq:plam_t_g_simplified} to \cref{eq:F0-line-deriv} gives:
  \begin{align*}
    \frac{d}{dt} \curlyb{F_0\parens{\lambda + t g; \theta}} &= \frac{l_\lambda}{s_\theta h} \parens{1 - t \frac{l_\lambda}{l_{\lambda + t g}}} \norm{\parens{\delP^\top}^\dagger \delU}^2_{\calP^\perp_{p_\lambda}}
  \end{align*}
  So, without loss of generality, if we set this derivative equal to
  zero and solve for $t$, it is sufficient to solve
  $l_{\lambda + t g} = t l_\lambda$ for $t$. Squaring both sides gives
  $Q(\lambda + t g) = t^2 Q(\lambda)$. Furthermore:
  \begin{align*}
    Q(\lambda + tg) = Q(\lambda) + 2tg^\top \delP^\top p_\lambda + t^2 g^\top \delP^\top \delP g = Q(\lambda) \squareb{1 + \frac{t^2}{{(s_\theta h)}^2} \norm{\parens{\delP^\top}^\dagger \delU}^2_{\calP^\perp_{p_\lambda}}}.
  \end{align*}
  Hence, since $Q(\lambda) > 0$, we get $1 + t^2 \beta_\lambda =
  t^2$. Solving for $t$ and noting that $t \leq 0$ is required gives
  us the result.
\end{proof}

\textbf{TODO}: it would be useful to prove a result like this where
the Hessian is perturbed by some $\epsilon \m{I}$ with $\epsilon > 0$.

\end{document}

%%% Local Variables:
%%% mode: latex
%%% TeX-master: t
%%% End:
