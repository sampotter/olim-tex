\documentclass[eikonal.tex]{subfiles}

\begin{document}

\section{Fast Projection}

\begin{defn}
  Let
  $\Delta^d = \set{(\lambda_1, \cdots, \lambda_d) \in \mathbb{R}^d :
    \lambda_i \geq 0 \mbox{ and } \sum_{i=1}^d \lambda_i \leq 1}$ be
  the \emph{barycentric coordinate simplex}.
\end{defn}

An algorithm for projecting points in $\R^d$ onto $\Delta^d$ in $O(d)$
time follows.

\begin{algorithm}
  \caption{projection alg.}
  \begin{algorithmic}
    \REQUIRE $y \in \R^d$
    \ENSURE $y \gets \arg\min_{x \in \Delta^d} \norm{y - x}_2$
    \STATE Set $\sigma \gets \sum_{i=1}^d y_i$
    \IF{$\sigma > 1$}
      \STATE Set $y \gets y + (1 - \sigma)/d$
    \ENDIF
    \STATE Set $y \gets \max(0, \min(y, 1))$
  \end{algorithmic}
\end{algorithm}

\begin{lemma}
  This algorithm is correct
\end{lemma}

\begin{proof}
  First, note that the column space of the matrix
  $U \in \R^{d \times d - 1}$ given by:
  \begin{align*}
    U = \begin{bmatrix} -\one_{1 \times d - 1} \\ I_{d-1} \end{bmatrix}
  \end{align*}
  is the dimension $d-1$ hyperplane
  $\{x \in \R^d : \one^\top x = 0\}$. Since the columns of $U$ are
  linearly independent, we can write the orthogonal projector onto
  this space as $UU^\dagger = U(U^\top U)^{-1}U^\top$. We have
  $U^\top U = I_{d-1} + \one_{d-1 \times d-1}$, and thus
  $(U^\top U)^{-1} = I_{d-1} - \one_{d-1 \times d-1}/d$, since
  (dropping subscripts, and temporarily writing $I = I_{d-1}$ and
  $\one = \one_{d-1 \times d-1}$):
  \begin{align*}
    \parens{I + \one} \parens{I - \tfrac{1}{d}\one} = I - \tfrac{1}{d} \one + \one - \tfrac{d-1}{d} \one = I.
  \end{align*}
  From this, we obtain $UU^\dagger = I_d - \one_{d \times d}/d$.

  Geometrically, we can see that an algorithm to orthogonally project
  $y$ onto $\Delta^d$ is as follows:
  \begin{enumerate}
  \item Let $\sigma = y_1 + \cdots + y_d$.
  \item If $\sigma > 1$, project $y$ onto the halfspace
    $\{x \in \R^d : \sum_{i=1}^d x_i \leq 1\}$.
  \item Clamp each $y_i$ to the range $0 \leq y_i \leq 1$.
  \end{enumerate}
  In homogeneous coordinates, the second step is equivalent to setting:
  \begin{align}\label{eq:proj-homogeneous}
    \begin{bmatrix}
      y \\ 1
    \end{bmatrix} \gets \begin{bmatrix}
      I & \one/d \\ & 1
    \end{bmatrix} \begin{bmatrix}
      UU^\dagger & \\
      & 1
    \end{bmatrix} \begin{bmatrix}
      I & -\one/d \\ & 1
    \end{bmatrix} \begin{bmatrix}
      y \\ 1
    \end{bmatrix}
  \end{align}
  if $\sigma > 1$. To see this, if $\sigma > 1$, then $y$ lie strictly
  outside of the halfspace of interest. Translation by $-\one/d$
  transforms the halfspace into
  $\set{x \in \R^d : \one^\top x \leq 0}$. Projection onto the plane
  $\set{x \in \R^d : \one^\top x = 0}$ is then equivalent to
  projection onto the transformed halfspace, and is effected by
  multiplication by $UU^\dagger$. Our matrix formulation simply
  carries this transformation out in homogeneous coordinates. If we
  multiply out \cref{eq:proj-homogeneous}, we see that this
  transformation is equivalent to:
  \begin{align*}
    \begin{bmatrix}
      y \\ 1
    \end{bmatrix} \gets \begin{bmatrix}
      I_d - \tfrac{1}{d} \one_{d \times d} & -\tfrac{1}{d} \one_{d \times 1} \\
      & 1
    \end{bmatrix} \begin{bmatrix}
      y \\ 1
    \end{bmatrix} = \begin{bmatrix}
      y + \frac{1 - \sigma}{d} \one_{d \times 1} \\
      1
    \end{bmatrix},
  \end{align*}
  i.e. $y \gets y + (1 - \sigma)/d$.
\end{proof}

\end{document}

%%% Local Variables:
%%% mode: latex
%%% TeX-master: t
%%% End:
