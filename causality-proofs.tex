\documentclass[sisc-eikonal.tex]{subfiles}

\begin{document}

\section{Proofs for section \ref{ssec:causality}}\label{sec:causality-proofs}

\begin{proof}[Proof of \cref{thm:causality}]
  For causality of $F_0$, we want $\hat{U} \geq \max_i U_i$, which is
  equivalent to $\min_i(\hat{U} - U_i) \geq 0$. From
  \cref{eq:U-finite-diff}, we have:
  \begin{equation}
    \min_i \parens{\hat{U} - U_i} = s^\theta h \min_i \min_{\lambda \in \Delta^n} \frac{\nu_i^\top \nu_{\lambda}}{\norm{p_\lambda}} = s^\theta h \min_{i, j} \frac{\nu_i^\top \nu_j}{\norm{p_i}} \geq 0.
  \end{equation}
  The last equality follows because minimizing the cosine between two
  unit vectors is equivalent to maximizing the angle between them;
  since $\lambda$ is restricted to lie in $\Delta^n$, this clearly
  happens at a vertex since the minimization problem is a linear
  program.

  For $F_1$, first rewrite $s_\lambda^\theta$ as follows:
  \begin{equation}
    s_\lambda^\theta = s^\theta + \theta (s_0 + \dels^\top \lambda - \overline{s}),
  \end{equation}
  where $\overline{s} = n^{-1} \sum_{i=0}^{n-1} s_i$. If
  $\lambda_0^\star$ and $\lambda_1^\star$ are the minimizing arguments
  for $F_0$ and $F_1$, respectively, and if
  $\dellam^* = \lambda_1^* - \lambda_0^*$, then we have:
  \begin{equation}\label{eq:F1-in-terms-of-F0}
    F_1^\theta(\lambda_1^*) = F_0^\theta(\lambda_1^*) + \theta \parens{s_0 + \dels^\top \lambda_1^* - \overline{s}} h l_{\lambda_1^\star}.
  \end{equation}
  By the optimality of $\lambda_0^*$ and strict convexity of
  $F_0^\theta$ (\cref{lemma:F-strictly-convex}), we can Taylor expand
  and write:
  \begin{equation}
    F_0^\theta(\lambda_1^*) = F_0^\theta(\lambda_0^*) + \nabla F_0^\theta(\lambda_0^*)^\top \dellam^* + \frac{1}{2} {\dellam^*}^\top \nabla^2 F_0^\theta(\lambda_0^*) \dellam^* + R \geq R,
  \end{equation}
  where $\abs{R} = O(h^3)$ by \cref{theorem:mp0-newton}. Let
  $\hat{U} = F_1^\theta(\lambda_1^*)$. Since $F_0^\theta$ is causal,
  we can write:
  \begin{equation}
    \hat{U} \geq \max_i U_i + R + \theta \parens{s_0 + \dels^\top \lambda_1^* - \overline{s}} h l_{\lambda_1^*}.
  \end{equation}
  Since $s$ is Lipschitz, the last term is $O(h^2)$---in particular,
  $\norm{\dels} = O(h)$ and $\norm{s_0 - \overline{s}} = O(h)$ since
  $s_0$ and $\overline{s}$ lie in the same simplex, whose diameter is
  $O(h)$. So, because the gap $\min_i(\hat{U} - U_i)$ is $O(h^2)$, we
  can see that $\hat{U} \geq \max_i U_i$ for $h$ sufficiently small.
\end{proof}

\end{document}

%%% Local Variables:
%%% mode: latex
%%% TeX-master: "sisc-eikonal"
%%% End:
