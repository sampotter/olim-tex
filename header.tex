\documentclass[eikonal.tex]{subfiles}

\begin{document}

\title{
  Ordered Line Integral Methods \\
  for Solving the Eikonal Equation
  \thanks{Submitted to the editors \today.}
  \funding{This work was partially supported by NSF Career Grant
    DMS1554907 and MTECH grant No.\ 6205.}
}
\author{
  Samuel F.\ Potter\thanks{Department of Computer Science,
    University of Maryland (\email{sfp@umiacs.umd.edu}).}
  \and Maria K.\ Cameron\thanks{Department of Mathematics,
    University of Maryland (\email{cameron@math.umd.edu}).}
}
\date{\today}

\maketitle

\begin{abstract}
  The eikonal equation is used to model high-frequency wave
  propagation and solve a variety of applied problems in computational
  science. We present a family of fast and accurate Dijkstra-like
  solvers for the eikonal equation and factored eikonal equation,
  which compute solutions on a regular grid by solving local
  variational minimization problems. Our methods converge linearly but
  compute significantly more accurate solutions than competing linear
  methods, due to improved directional coverage and the use of more
  accurate quadrature rules. In 3D, we present two different families
  of algorithms which significantly reduce the number of FLOPs needed
  to obtain an accurate solution to the eikonal equation. One method
  employs a fast search using local characteristic directions to prune
  unnecessary updates, and the other uses the theory of constrained
  optimization to achieve the same end. The proposed solvers are more
  efficient than the standard fast marching method in terms of the
  relationship between error and CPU time. We also modify our method
  for use with the additively factored eikonal equation, which can be
  solved locally around point sources to maintain linear
  convergence. We conduct extensive numerical simulations and provide
  theoretical justification for our approach. A library that
  implements the proposed solvers is available online.
\end{abstract}

\begin{keywords}
  ordered line integral method, eikonal equation, factored eikonal
  equation, simplified midpoint rule, semi-Lagrangian method, fast
  marching method
\end{keywords}

\begin{AMS}
  65N99, 65Y20, 49M99
\end{AMS}

\end{document}

%%% Local Variables:
%%% mode: latex
%%% TeX-master: "sisc-eikonal.tex"
%%% End:
