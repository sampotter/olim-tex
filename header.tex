\documentclass[eikonal.tex]{subfiles}

\begin{document}

\title{Ordered Line Integral Methods \\ for Solving the Eikonal Equation}
\author{Samuel F.\ Potter and Maria K.\ Cameron}
\date{\today}

\maketitle

\begin{abstract}
  The eikonal equation is used to model high-frequency wave
  propagation and solve a variety of applied problems in computational
  science. We present a fast and accurate Dijkstra-like solver for the
  eikonal equation and factored eikonal equation, which computes
  solutions on a regular grid in 2D and 3D by solving local raytracing
  problems. Our method converges linearly but computes significantly
  more accurate solutions than competing methods. We consider a
  variety of quadrature rules for approximately solving local
  raytracing problems. In 3D, we consider two different hierarchical
  algorithms which significantly reduce the number of FLOPs needed to
  solve the eikonal equation. One method employs a fast neighborhood
  search to prune unnecessary updates, and the other uses
  Karush-Kuhn-Tucker theory to achieve the same end. We conduct
  extensive numerical simulations and provide theoretical
  justification for our approach. A link to a library implementing
  these algorithms is available online at the authors' respective
  webpages.
\end{abstract}

\begin{keywords}
  eikonal equation, factored eikonal equation, fast marching method,
  semi-Lagrangian method
\end{keywords}

% https://mathscinet.ams.org/msc/pdfs/classifications2010.pdf
\begin{AMS}
  97N40, \hl{\textbf{TODO}}: \emph{add more from
    \href{https://mathscinet.ams.org/msc/pdfs/classifications2010.pdf}{here}.} % numerical analysis
\end{AMS}

\end{document}

%%% Local Variables:
%%% mode: latex
%%% TeX-master: "sisc-eikonal.tex"
%%% End:
