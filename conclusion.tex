\documentclass[eikonal.tex]{subfiles}

\begin{document}

\section{Conclusion}

We have presented two families of fast and accurate direct solvers for
the eikonal equation. One of these relies on enumerating valid update
simplexes, while the other employs a fast search for the first arrival
characteristic. These methods use different quadrature rules: a
simplified midpoint rule (\texttt{mp0}), a midpoint rule
(\texttt{mp1}), and a righthand rule (\texttt{rhr}). We analyze the
relationship between these quadrature rules and prove error bounds. We
conduct numerical experiments measuring runtime and relative
$\ell_\infty$ error. We make a comparison with the standard fast
marching method which is equivalent to our \texttt{olim6\_rhr},
showing that they are comparable in terms of speed and accuracy, with
\texttt{olim6\_rhr} being slightly slower. To determine the relative
time spent on different tasks, we profile our C++ implementation using
Valgrind, separating time spent into several coarse-grained
categories. From this, we show that for practical problem sizes, the
runtime of Dijkstra-like algorithms behaves like $C N^n$, where
$n = 2, 3$, and $N^n$ is the total number of gridpoints (even if this
is not strictly true from a computational complexity viewpoint); we
also emphasize that memory access patterns play a large role in
algorithm runtime, especially for large $N$.

Overall, we conclude that ordered line integral methods are a powerful
approach to obtaining a higher degree of accuracy when solving the
eikonal equation in 3D. With an appropriate choice of quadrature rule,
we are able to exploit improved directional coverage to drive down the
error constant. The improved accuracy more than makes up for the
modest price paid in speed, and we fully expect it to be possible to
find ways to optimize this family of algorithms further. We have also
tried to demonstrate that memory access patterns dominate both update
time and time spent maintaining the front data structure, from which
we can conclude two things: 1) the exact time spent updating a node is
important but not paramount (improving accuracy is more important than
improving speed), 2) using memory optimally will lead to a substantial
speed-up for large problems.

\end{document}

%%% Local Variables:
%%% mode: latex
%%% TeX-master: "sisc-eikonal.tex"
%%% End:
