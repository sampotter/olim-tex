\documentclass[10pt]{article}

\usepackage{amsthm}
\usepackage{eikonal}
\usepackage{subfiles}

\theoremstyle{plain}
\newtheorem{corollary}{Corollary}
\newtheorem{lemma}{Lemma}
\newtheorem{theorem}{Theorem}

\theoremstyle{definition}
\newtheorem{defn}{Definition}

\theoremstyle{remark}
\newtheorem{remark}{Remark}

\begin{document}

\title{Ordered Line Integral Methods \\ for Solving the Eikonal Equation}
\author{Samuel F. Potter \and Maria Cameron}
\date{\today}

\maketitle

\begin{abstract}
  \textbf{TODO}: \hl{write this}.
\end{abstract}

\subfile{introduction.tex}
\subfile{background.tex}
\subfile{preliminaries.tex}
\subfile{quadrature.tex}
\subfile{olims.tex}
\subfile{f0-exact.tex}
\subfile{mp0_vs_mp1.tex}
\subfile{constrained-newton.tex}
% \subfile{unconstrained-newton.tex}
\subfile{kkt-hu.tex}
\subfile{tetrahedron-enumeration.tex}
% \subfile{olim-types.tex}
% \subfile{olim3d-newton.tex}
% \subfile{mp1-tri-newton.tex}
% \subfile{skipping.tex}

% \appendix

% \subfile{fast-simplex-projection.tex}
% \subfile{simplified-updates.tex}

\bibliographystyle{plain}
\bibliography{eikonal}{}

\end{document}

%%% Local Variables:
%%% mode: latex
%%% TeX-master: t
%%% End:
