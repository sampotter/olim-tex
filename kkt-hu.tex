\documentclass[eikonal.tex]{subfiles}

\begin{document}

\section{A Hierarchical Update}

\begin{itemize}
\item If we think of ourselves as tracking a single characteristic and
  finding the location where it arrives (minimizes our cost
  functional), then we can think of ordering our updates in a way that
  allows us to find a reasonably good approximation to the minimizer
  by performing fewer updates
\item We can optimize the foregoing approach by taking advantage of
  KKT theory to skip costly higher-dimensional updates (tetrahedron
  updates, in particular)
\end{itemize}
The algorithm for updating a new trial node $\hat{p}$ is roughly like this:

\begin{mdframed}
  \begin{enumerate}[nolistsep]
  \item Let $p_0 \in \texttt{nb}(\hat{p})$ be the node such that
    $p\texttt{.state} = \texttt{valid}$ with the minimal degree 1 update
    value \label{item:find-p0}
  \item For each $p_1 \in \texttt{nb}(\hat{p})$ such that
    $p_1\texttt{.state} = \texttt{valid}$ and
    $\norm{p_1 - p_0} \leq \sqrt{3}$, consider the triangle update
    with base $[p_0, p_1]$:
    \begin{enumerate}[nolistsep]
    \item Use KKT theory to check if $p_0$ or $p_1$ are the global
      minimizer over $[p_0, p_1]$\label{item:kkt-tri}
    \item If they aren't, perform the triangle
      update\label{item:p1-tri-update}
    \end{enumerate}\label{item:find-p1}
  \item Let $p_1$ be the node in \cref{item:find-p1} which corresponds
    to the minimum value triangle update in \cref{item:p1-tri-update}
  \item For each $p_2 \in \texttt{nb}(\hat{p})$ such that
    $p_2\texttt{.state} = \texttt{valid}$, and with at most one of the
    pairwise distances between $p_0, p_1$, and $p_2$ being in the
    range $(\sqrt{2}, \sqrt{3}]$ (with the others at most $\sqrt{2}$),
    consider the tetrahedron update with characteristic emanating from
    $\conv(\{p_0, p_1, p_2\})$:
    \begin{enumerate}[nolistsep]
    \item Use KKT theory check if the update needs to be performed:
      \begin{enumerate}[nolistsep]
      \item If necessary, compute the triangle updates corresponding
        to $[p_0, p_2]$ and $[p_1, p_2]$
      \item Using KKT theory, check if any of the minimizers of the
        triangle updates corresponding to $[p_0, p_1], [p_0, p_2]$, or
        $[p_1, p_2]$ are the global minimizers over
        $\conv(\{p_0, p_1, p_2\})$\label{item:kkt-tetra}
      \end{enumerate}
    \item If necessary, perform the update
    \end{enumerate}
  \end{enumerate}
\end{mdframed}

\noindent Some caveats and remarks:
\begin{itemize}[nolistsep]
\item The choice of $p_0$ in \cref{item:find-p0} may not be unique;
  or, there may be several candidates with solutions that are
  extremely close. This should happen in the case of multiply arriving
  characteristics. \textbf{TODO}: what should we say about this?
\item The restrictions on $p_0$, $p_1$, and $p_2$ are such that once
  $p_0$ and $p_1$ have been fixed, $p_2$ will lie in the same orthant
  as $p_0$ and $p_1$. This can help us simplify our enumeration of
  tetrahedra
\item It should be possible to extend this straightforwardly to grids
  in $\mathbb{R}^n$
\item It seems possible to do a complexity analysis and show that the
  update procedure should only take $O(n^3)$ FLOPs, where $n$ is the
  dimension (e.g., for Newton's method, the number of iterations
  should be constant with respect to $n$)
\end{itemize}

\paragraph{Skipping higher-degree updates.} This section will expand
on the details of \cref{item:kkt-tri,item:kkt-tetra}. Consider the
problem:
\begin{align*}
  \min_{\lambda} &\qquad F_i(\lambda_i) \\
  \text{subject to} &\qquad \lambda \in \Delta^n
\end{align*}
This is a nonlinear program with linear inequality constraints. If
$i = 0$, the problem is strictly convex; the same is true if $i = 1$
and $h > 0$ is small enough. From here on, we will drop the subscript
$i$ and write $F = F_i$ for simplicity. If we fix
$\lambda \in \Delta^n$ and linearize the problem, we obtain the
following inequality-constrained quadratic program:
\begin{align*}
  \min_{\delta\lambda} &\qquad \frac{1}{2} \delta\lambda^\top \nabla^2 F(\lambda) \delta\lambda + \nabla F(\lambda)^\top \delta\lambda \\
  \text{subject to} &\qquad \lambda + \delta\lambda \in \Delta^n
\end{align*}
The constraint $\lambda + \delta\lambda \in \Delta^n$ can be written
$A\delta\lambda \leq b - A\lambda$. For our fixed $\lambda$, if we let
$I$ denote the set of active constraints, and let $A_j$ be the $j$th
row of $A$, then the local problem corresponding to the active
constraints is:
\begin{align*}
  \min_{\delta\lambda} &\qquad \frac{1}{2} \delta\lambda^\top \nabla^2 F(\lambda) \delta\lambda + \nabla F(\lambda)^\top \delta\lambda \\
  \text{subject to} &\qquad A_I \delta\lambda = b_I - A_I\lambda
\end{align*}
The KKT system corresponding to this quadratic program is:
\begin{align*}
  \begin{bmatrix}
    \nabla^2 F(\lambda) & -A_I^\top \\
    A_I & 0
  \end{bmatrix} \begin{bmatrix}
    \delta \lambda^* \\
    \mu^*
  \end{bmatrix} = \begin{bmatrix}
    -\nabla F(\lambda) \\
    b_I - A_I\lambda
  \end{bmatrix} = \begin{bmatrix}
    -\nabla F(\lambda) \\
    0
  \end{bmatrix},
\end{align*}
where $\mu^*$ is the vector of Lagrange multipliers, and noting that
$b_I - A_I\lambda = 0$ by definition of $I$. If we multiply the first
equation by $A_I \nabla^2 F(\lambda)^{-1}$, matching the result with
the second equation allows us to solve for $\mu^*$ directly, giving:
\begin{equation}\label{eq:mu}
  \mu^* = {(A_I \nabla^2 F(\lambda)^{-1} A_I^\top)}^{-1} A_I \nabla^2 F(\lambda)^{-1} \nabla F(\lambda).
\end{equation}
Computing $\mu^*$ from \cref{eq:mu} may appear onerous, but the matrix
$A_I$ has a simple, sparse form, and conjugation by $A_I$ has the
effect of reducing the dimensionality of the problem.

Taking a tetrahedron update as an example, once we have access to
$\mu^*$ for each triangle update corresponding to the boundary of the
base of the update simplex, we can look at $\mu^*$ for the minimal
triangle update. Since $\nabla^2 F$ is positive definite (again, for
$h$ small enough), if $\mu^* < 0$, then the necessary conditions for
optimality are violated, and the unconstrained minimum of $F$ must lie
outside of $\Delta^2$; hence, the corresponding $\lambda^*$ is the
global minimum of the constrained problem. On the other hand, if
$\mu^* \geq 0$, since the minima of the other two triangle updates are
both greater than the minimum for the triangle update under
consideration, we can conclude that the constrained optimum must lie
in $\Delta^2$. (\emph{Terribly worded, but I think this is the
  argument...})

When attempting to skip a triangle update, the boundary of $\Delta^1$
is the set $\{0\} \cup \{1\}$. In this case, we can verify that the
preceding check simplifies to looking at the signs of $F'(0)$ and
$F'(1)$.

\paragraph{A simplified rule for skipping triangle updates.} For a
triangle update, observe that $\Delta^1$ is the set of $\lambda$ such
that $A\lambda \geq b$, where:
\begin{equation}
  A = \begin{bmatrix} 1 \\ -1 \end{bmatrix}, \qquad b = \begin{bmatrix} 0 \\ -1 \end{bmatrix},
\end{equation}
i.e. $\Delta^1 = [0, 1]$; hence, the set of active indices $I$ can
only equal $\set{0}$ or $\set{1}$, corresponding to $\lambda = 0$ or
$\lambda = 1$, respectively. Let $\mu_0$ be the Lagrange multiplier
corresponding to $\lambda = 0$; likewise, define $\mu_1$ for
$\lambda = 1$. The vector $b$ has no bearing on computing $\mu_0$ or
$\mu_1$. If we let $i = 0, 1$, $\lambda = i$, and $I = \set{i}$, we
can see that $A_I = (-1)^i$ so that \cref{eq:mu} simplifies to:
\begin{equation}
  \mu_i = \parens{\frac{{(-1)}^{2i}}{F''(\lambda)}}^{-1} {(-1)}^{i} \frac{F'(\lambda)}{F''(\lambda)} = \begin{cases}
    F'(0) & \mbox{if } i = 0, \\
    -F'(1) & \mbox{if } i = 1.
  \end{cases}
\end{equation}
Since we skip if $\mu > 0$, we obtain the following simple and
intuitive rule for skipping triangle updates.
\begin{mdframed}
  \begin{center}
    For a triangle update, if either $F'(0) > 0$ or $F'(1) < 0$, skip
    the update.
  \end{center}
\end{mdframed}
Unlike triangle updates, tetrahedron updates becomes more involved
since a matrix must actually be inverted. It may be possible to obtain
a simplified form for $\mu$ by using the Woodbury formula, but it
seems unlikely to be that helpful for $n = 2$ since evaluating
\cref{eq:mu} directly is cheap once $\nabla^2 F(\lambda)^{-1}$ is
available. For larger values of $n$, it may be worth pursuing.

\end{document}

%%% Local Variables:
%%% mode: latex
%%% TeX-master: t
%%% End:
