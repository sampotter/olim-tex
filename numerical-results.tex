\documentclass[sisc-eikonal.tex]{subfiles}

\begin{document}

\section{Numerical Results}\label{sec:numerical-results}

\subsection[Single point source]{Nonsymmetric slowness functions with
  a single point source}\label{ssec:point-source-problems}

Using \cref{eq:eikonal}, a simple recipe to create pairs of slowness
functions $s$ and solutions $u$ is to prescribe a solution $u$ and
compute $s(x) = \norm{\nabla u(x)}_2$. In general, $u$ may be
discontinuous due to rarefaction fans and shocks in the solution, so
some care must be taken with this. Although it's possible to use this
recipe with complicated choices of $\boundary$, we limit ourselves to
considering a single point source at the origin---i.e.,
$\boundary = \set{0}$.

\begin{center}
  \begin{tabular}{cccc}
    Name & $u(x)$ & $s(x)$ \\
    \midrule
    \texttt{s1} & $\cos(r) + r - 1$ & $1 - \sin(r)$ \\
    \texttt{s4} & $\tfrac{1}{2} x^\top A^{1/2} x$, where $A^\top = A$ and $A \succ 0$ & $\norm{x}_A$
  \end{tabular}
\end{center}

A simple recipe for generating \texttt{s4} is to compute a random
orthogonal matrix $Q$ using the QR decomposition, letting
$\Lambda = \operatorname{diag}(\lambda_i)$ where $\lambda_i > 0$ be a
diagonal matrix, and setting $A = Q\Lambda Q^\top$.

\subsection{Slotnick's slowness function}\label{ssec:slotnick}

We consider a problem that has an analytical ray tracing solution and
is used as a test problem elsewhere for solvers for the factored
eikonal
equation~\cite{slotnick1959lessons,fomel2009fast,qi2018corner}. For a
single point source at $x_i$ and a vector $v$, we define:
\begin{equation}
  \label{eq:slotnick-single-source}
  \frac{1}{s_i(x)} = \frac{1}{s_i} + v^\top {(x - x_i)},
\end{equation}
hence $s_i(x_i) = s_i$. The analytic solution to \cref{eq:eikonal} for a
single source and slowness function given by
\cref{eq:slotnick-single-source} is:
\begin{equation}
  \label{eq:slotnick-single-source-solution}
  u_i(x) = \frac{1}{\norm{v}} \cosh^{-1} \parens{1 + \frac{s_i}{2} s(x) \norm{v}^2 \norm{x - x_i}^2}.
\end{equation}
If we shift the point source from $x_i$ to another location $x_j$, we
find:
\begin{equation}
  \label{eq:slotnick-slowness-shift}
  \frac{1}{s_i(x)} = \frac{1}{s_i} + v^\top {(x - x_j + x_j - x_i)} = \frac{1}{s_i} + v^\top {(x_j - x_i)} + v^\top {(x - x_j)} = \frac{1}{s_i(x_j)} + v^\top {(x - x_j)}.
\end{equation}
Defining $s_j = s_i(x_j)$ and $s_j(x)$ from:
\begin{equation}
  \frac{1}{s_j(x)} = \frac{1}{s_j} + v^\top {(x - x_j)},
\end{equation}
we can see that $s_j(x_i) = s_i(x_j)$, that shifting the point source
from $x_i$ to $x_j$ is equivalent to changing the parameter $s_i$ to
$s_j$, and that $u_j(x)$ is defined analogously to
\cref{eq:slotnick-single-source-solution}. The solution for multiple
point sources $\set{x_i}$ is then given by:
\begin{equation}
  u(x) = \min_i u_i(x).
\end{equation}

\subsection{Comparison of hierarchical algorithms}\label{ssec:alg-comparison}

\Cref{alg:bottom-up,alg:top-down} can be compared in terms of the
number of updates of different degree that they
perform. In~\cref{tab:stats}, for our usual single point source
problem on $\Omega = [-1, 1]^3$, discretized into a grid with $n^3$
points, where $n = 5, 9, \hdots, 129$, we \hl{\textbf{TODO}}:
\emph{explain what we do...}

\begin{table}
  \centering
  \begin{tabular}{c|r|r|r|r|r}
& $N$ & Avg. Visits & $d = 0$ & $d = 1$ & $d = 2$ \\
\midrule
\multirow{6}{*}{\texttt{olim18\_rhr}} & 5 & 6.2400 & 1.0000 & 1.2481 & 0.1916 \\
& 9 & 7.4074 & 1.0000 & 1.2526 & 0.2738 \\
& 17 & 8.1384 & 1.0000 & 1.2426 & 0.3187 \\
& 33 & 8.5510 & 1.0000 & 1.2338 & 0.3477 \\
& 65 & 8.7707 & 1.0000 & 1.2284 & 0.3624 \\
& 129 & 8.8841 & 1.0000 & 1.2254 & 0.3702 \\
\midrule
\multirow{6}{*}{\texttt{olim26\_rhr}} & 5 & 8.2880 & 1.0000 & 0.5453 & 0.0962 \\
& 9 & 10.2167 & 1.0000 & 0.5202 & 0.1843 \\
& 17 & 11.4732 & 1.0000 & 0.4930 & 0.2382 \\
& 33 & 12.1982 & 1.0000 & 0.4774 & 0.2638 \\
& 65 & 12.5889 & 1.0000 & 0.4696 & 0.2762 \\
& 129 & 12.7918 & 1.0000 & 0.4656 & 0.2822 \\
\midrule
\multirow{6}{*}{\texttt{olimhu\_rhr}} & 5 & 8.2880 & 1.0000 & 1.1886 & 0.6547 \\
& 9 & 10.2167 & 1.0000 & 1.3074 & 0.8064 \\
& 17 & 11.4732 & 1.0000 & 1.3629 & 0.8626 \\
& 33 & 12.1982 & 1.0000 & 1.3890 & 0.8773 \\
& 65 & 12.5889 & 1.0000 & 1.4022 & 0.9062 \\
& 129 & 12.7918 & 1.0000 & 1.4092 & 0.9099 \\
\midrule
\multirow{6}{*}{\texttt{olim6\_rhr}} & 5 & 2.4000 & 1.0000 & 1.0753 & 0.1720 \\
& 9 & 2.6667 & 1.0000 & 1.2015 & 0.2344 \\
& 17 & 2.8235 & 1.0000 & 1.2682 & 0.2780 \\
& 33 & 2.9091 & 1.0000 & 1.3013 & 0.3039 \\
& 65 & 2.9538 & 1.0000 & 1.3175 & 0.3182 \\
& 129 & 2.9767 & 1.0000 & 1.3255 & 0.3256 \\
\midrule
\end{tabular}

  \caption{Table of update statistics for~\cref{ssec:alg-comparison}.}
  \label{tab:stats}
\end{table}

\end{document}

%%% Local Variables:
%%% mode: latex
%%% TeX-master: "sisc-eikonal.tex"
%%% End:
