\documentclass[12pt]{article}

\usepackage{amsmath}
\usepackage{amssymb}
\usepackage{amsthm}
\usepackage{fullpage}
\usepackage{graphicx}
\usepackage{sansmathfonts}
\usepackage[T1]{fontenc}
\renewcommand*\familydefault{\sfdefault} %% Only if the base font of the document is to be sans serif
\usepackage{hyperref}

\newcommand{\curlyb}[1]{\left\{#1\right\}}
\newcommand{\norm}[1]{\left\|#1\right\|}
\newcommand{\set}[1]{\curlyb{#1}}

\DeclareMathOperator{\Jac}{Jac}

\title{Solving lifted anisotropic eikonal equations}
\author{Samuel F. Potter\footnote{Homepage:
    \url{https://umiacs.umd.edu/\textasciitilde sfp}, email:
    \url{sfp@umiacs.umd.edu}.}}
\date{\today}

\begin{document}

\maketitle

\section{Lifted Hamilton-Jacobi?}

We'll start by considering an anisotropic eikonal equation on a
regular square grid in $\mathbb{R}^2$. Without loss of generality, let
$\Omega = [0, 1]^2$ and define:
\begin{equation}
  \mathcal{G} = \set{(u_i, v_j) = (hi, hj) : 0 \leq i \leq N, 0 \leq j
    \leq N} \subseteq \Omega,
\end{equation}
where $N$ is a positive integer and $h = N^{-1}$. Now, let
$A(u, v) : \mathbb{R}^2 \to \mathbb{R}^{2 \times 2}$ be a matrix field
where $A(u, v)$ is symmetric positive definite for all
$(u, v) \in \Omega$. For each $(u, v) \in \Omega$,
\begin{equation}
  \langle \mu, \nu \rangle_{A(u, v)} = \mu^\top A(u, v) \nu
\end{equation}
is an inner product, and
$\norm{\mu}_{A(u, v)} = \sqrt{\langle \mu, \mu \rangle_{A(u, v)}}$ is
a norm. For a nonnegative scalar field
$\sigma : \Omega \to \mathbb{R}$, the anisotropic eikonal equation is:
\begin{equation}
  \norm{\nabla \sigma(u, v)}_{A(u, v)} = 1.
\end{equation}
To specify boundary conditions, we assume there is a subset
$\mathcal{B} \subseteq \Omega$ where $\sigma$ agrees with boundary
data specified by a nonnegative function
$b : \mathcal{B} \to \mathbb{R}$:
\begin{equation}
  \left. \sigma(u, v) \right|_{\mathcal{B}} = b(u, v).
\end{equation}

Our goal is to find a surface whose first fundamental form agrees with
the spatially varying metric $A(u, v)$ and solve an isotropic eikonal
equation on that surface. Let $X : \Omega \to \mathbb{R}^3$
parametrize a surface. The first fundamental form $I$ of the surface
$S = X(\Omega)$ at a point $X(u, v)$ is given by:
\begin{equation}
  I = \begin{bmatrix}
    \langle X_u, X_u \rangle & \langle X_u, X_v \rangle \\
    \langle X_v, X_u \rangle & \langle X_v, X_v \rangle
  \end{bmatrix} = \begin{bmatrix}
    X_u & X_v
  \end{bmatrix}^\top \begin{bmatrix}
    X_u & X_v
  \end{bmatrix} = \Jac_X(u, v)^\top \Jac_X(u, v).
\end{equation}
To accomplish our task, we need to compute $X$ such that:
\begin{equation}
  A(u, v) = \Jac_X(u, v)^\top \Jac_X(u, v)
\end{equation}
for all $(u, v) \in \Omega$. The particular conditions that need to be
satisfied are:
\begin{equation}
  \norm{X_u}_2^2 = A_{11}, \hspace{1em} \langle X_u, X_v \rangle = A_{12}, \hspace{1em} \norm{X_v}_2^2 = A_{22}.
\end{equation}
Our approach will be to compute a piecewise polynomial interpolant
(probably a $B$-spline surface) and attempt to stipulate that these
conditions hold and then solve for the corresponding set of polynomial
coefficients. In the process, we should recover a surface which we can
then solve an eikonal equation on.

\section{Alternative: using the gradient theorem}

\paragraph{Another idea.} Let's assume that we have a tensor field
$A : \mathbb{R}^2 \to \mathbb{R}^{2 \times 2}$, so that $A$ is spd
everywhere. If, for $(x, y)$ fixed:
\begin{equation}
  A = \begin{bmatrix}
    a & b \\ b & c
  \end{bmatrix},
\end{equation}
then the LDL decomposition of $A$ can be written:
\begin{equation}
  A = \begin{bmatrix}
    1 & \\ a^{-1}b & 1
  \end{bmatrix} \begin{bmatrix}
    a & \\ & c - a^{-1} b^2
  \end{bmatrix} \begin{bmatrix}
    1 & a^{-1} b \\ & 1
  \end{bmatrix},
\end{equation}
so that lower triangular factor of the Cholesky factorization of $A$
is just:
\begin{equation}
  L = \begin{bmatrix}
    1 & \\ a^{-1} b & 1
  \end{bmatrix} \begin{bmatrix}
    \sqrt{a} & \\ & \sqrt{c - a^{-1}b^2}
  \end{bmatrix} = \begin{bmatrix}
    \sqrt{a} & \\ \frac{b}{\sqrt{a}} & \frac{b}{a} \sqrt{c - \frac{b^2}{a}}
  \end{bmatrix}
\end{equation}
Provided that $A(x, y)$ is continuous, $L = L(x, y)$ is also
continuous and well-defined since $A$ is positive definite ($L$ is
also unique).

Now, if we would like to solve the anisotropic eikonal equation:
\begin{equation}
  \norm{\nabla u}_A = 1
\end{equation}
if we define a new scalar field $v$ such that
$\nabla v = L^\top \nabla u$, then:
\begin{equation}
  \nabla u^\top A \nabla u = \nabla u^\top L L^\top \nabla u = \nabla v^\top \nabla v.
\end{equation}
Hence, $\norm{\nabla u}_A = \norm{\nabla v}$. For a point source in a
convex domain $\Omega$, it seems clear that if we first solve
$\norm{\nabla v} = 1$, we can then use the gradient theorem to
evaluate $u$ in $\Omega$; i.e., we compute solve
$L^\top \nabla u = \nabla v$ on $\Omega$ to recover $\nabla u$
everywhere, and then use the fact that:
\begin{equation}
  u_1 - u_0 = \int_\gamma \nabla u^\top d\gamma
\end{equation}
to evaluate $u$ throughout $\Omega$.

Some observations:
\begin{itemize}
\item The accuracy is probably going to be limited by where we're able
  to compute $\nabla u$. We have to approximate the integral
  $\int_\gamma \nabla u^\top d\gamma$ with the data available to us,
  which will limit the accuracy of $u$.
\item In principle, we can compute all of the integrals
  $\int_\gamma \nabla u^\top d\gamma$ simultaneously and in parallel
  (and in $O(N)$ time).
\item Computing $\nabla v$ is also easy for a point source in a convex
  domain. If $(x_0, y_0) \in \Omega$ is the location of the point
  source, then $v(x, y) = \norm{(x - x_0, y - y_0)}$, and:
  \begin{equation}
    \nabla v(x, y) = \frac{1}{v(x, y)} \begin{bmatrix}
      x - x_0 \\ y - y_0
    \end{bmatrix}
  \end{equation}
\end{itemize}

\paragraph{A test problem.} The simplest thing to do would be to solve
the eikonal equation with $s \equiv 1$ in $\Omega = [0, 1]^d$, where
$d = 2, 3$. In this case, the characteristics behave predictably, and
we can see how the discretization of
$\int_\gamma \nabla u^\top d\gamma$ affects the accuracy. Furthermore,
we can see how choosing the ``summation graph'' affects the
accuracy. It's the least clear from the formulation given above what
should be done here. It might be that we end up back where we started
with the OLIM. For parallelization, if we can get away with summing
along axes of the standard grid, then it seems likely that we can
think of $u$ itself as a sort of ``summed area table'' and take
advantage of parallelization techniques related to those.


\end{document}

%%% Local Variables:
%%% mode: latex
%%% TeX-master: t
%%% End:
